%!TEX TS-program = pdflatex
%!TEX root = main.tex
%!TEX encoding = UTF-8 Unicode

\section{Minizinc: 005 vs 008}
\todo[inline]{
OCCHIO!
Questa sezione è una memo per me.
Probabilmente NON verrà messa nella relazione finale
}\noindent
I due modelli sono molto diversi:
il primo si basa su una grossa matrice 3d in cui ogni cella rappresenta una stanza e mentre l'intero salvato nella cella rappresenta il tipo di ospite per quella stanza.
Il secondo modello sfrutta 4 array, la cui lunghezza dipende dal numero di persone che dobbiamo piazzare, e i valori assegnati alle celle degli array corrispondono al numero della stanza.
La formalizzazione dei vincoli è abbastanza diversa per le due codifiche.
Infatti, nella prima i vincoli di vicinanza dipendono dalle coordinate di accesso alla matrice 3d, mentre nel secondo dipendono dai valori salvati negli array.
Nel secondo modello risulta molto più facile scrivere vincoli spezza simmetrie, ad esempio l'alldifferent o il vincolo id ordinamento crescente sugli array.

Seguono una descrizione più dettagliata dei modelli ed un loro confronto.

\subsection{005}

\lstinputlisting[firstline=47,lastline=59]{code/005_covid19.mzn}


