\documentclass{article}
%\documentclass[draft]{article}

\usepackage[italian]{babel}
\usepackage[utf8]{inputenc}


\usepackage{graphicx} % Immagini fantastiche e...
\graphicspath{        % dove trovarle
  {./images/},
  {../images/}        % Necessario se cartella chapters
}

% Bibliografia
\usepackage{csquotes}
\usepackage[
backend=biber, % backend... boh
style=numeric, % altri stili https://www.overleaf.com/learn/latex/Biblatex_bibliography_styles
sorting=nty % name, title, year % ynt % tny
]{biblatex}
\addbibresource{bibl.bib} % Il file della bibliografia

% Per blocchi di codice
\usepackage{listings}
\lstset{ % per apici e pedici nel codice
  mathescape
}

% Grafici direttamente in latex
\usepackage{tikz}
\usetikzlibrary{shapes,positioning,calc}
\colorlet{lightgray}{gray!20}

% Immagini galleggiano
%\usepackage{float}

\usepackage{caption}
% per caption ad immagini in tab annidiate
\usepackage{subcaption}

\usepackage{hyperref} % lasciare per ultimo
%\hypersetup{colorlinks=true, linkcolor=blue, citecolor=black, plainpages=false, urlcolor=blue}
\hypersetup{colorlinks=true, linkcolor=black, citecolor=black, plainpages=false, urlcolor=blue}

% Usato nella copertina
\usepackage{wallpaper}


% Esempio di impostazione per listings
\lstset{  %TODO migliorare stile
   language=SQL,
   showspaces=false,
   showstringspaces=false,
   basicstyle=\ttfamily,
%   %numbers=left,
%   %numberstyle=\tiny,
   commentstyle=\color{gray}
 }

% Simboli matematici
%\usepackage{amsthm}
\usepackage{amsfonts}
\usepackage{mathabx} % per \topdoteq
%\usepackage{mathtools, amsthm}

% Comodo per evidenziare zone da modificare
\usepackage{todonotes} 

% Lorem ipsum...
\usepackage{lipsum} 


 
% ================================ %
%          Cose Personali          %
% ================================ %
\newtheorem{theorem}{Theorem}[section]
\newtheorem{corollary}{Corollary}[theorem]
\newtheorem{lemma}[theorem]{Lemma}

% DSD Syntax Symbols
% DA USARE DENTRO MATH ENVIRONMENT
\newcommand{\toe}{^\wedge}
\newcommand{\cmpl}{^*}
\newcommand{\toec}{^{\wedge *}}

% Alcune comodita' logico/matematiche
\let\implies\rightarrow
\let\Implies\Rightarrow
\let\impliedby\leftarrow
\let\Impliedby\Leftarrow
\let\iff\Leftrightarrow

\let\et\wedge
\let\vel\vee

\newcommand\N{\ensuremath{\mathbb{N}}}
\newcommand\R{\ensuremath{\mathbb{R}}}
\newcommand\Z{\ensuremath{\mathbb{Z}}}
\renewcommand\O{\ensuremath{\emptyset}}
\newcommand\Q{\ensuremath{\mathbb{Q}}}
\newcommand\C{\ensuremath{\mathbb{C}}}

% DA USARE DENTRO NEL TESTO
% per evitare di scrivere "emph"
% notare spazio finale
\newcommand{\pattern}{\emph{pattern} }
\newcommand{\toehold}{\emph{toehold} }
\newcommand{\bind}{\emph{bind} }
\newcommand{\unbind}{\emph{unbind} }
\newcommand{\displacement}{\emph{displacement} }
\newcommand{\bond}{\emph{bond} }
\newcommand{\sg}{\emph{strand graph} }
\newcommand{\DNAcomp}{ \emph{DNA-Computing} }
\newcommand{\foot} {\emph{foot} }
\newcommand{\leg}  {\emph{leg}  }
\newcommand{\hand} {\emph{hand} }
\newcommand{\arm}  {\emph{arm}  }
% vari
%\newcommand{\lsti{args}}
%           {\lstinline{args}} % TODO cercare sintassi args



% ================================ %
%            Il Documento          %
% ================================ %
\begin{document}

% ================================ %
%        Creo Prima Pagina         %
% ================================ %

\ThisCenterWallPaper{0.95}{polloPallido}
% Intestazione
% TODO Migliorare esteticamente
\begin{titlepage}
 	\centering
  \Huge{\textbf{Approfondimento per\\ Bella Materia}}\\
 	[25mm]
  \raggedright
  \Large{\textbf{Studente:}}\\
  \Large{\textbf{Tristano Munini}}\\
 	[25mm]
  \raggedright
  \Large{\textbf{Relatore:}}\\
  \Large{\textbf{Prof. Professore}}\\
 	[80mm]
 	\centering
  \LARGE{\underline{\textbf{ANNO ACCADEMICO 2019-2020}}}\\
\end{titlepage}

% ================================ %
%              Indice              %
% ================================ %
\tableofcontents
\thispagestyle{empty}
\cleardoublepage
\setcounter{page}{1}


% ================================ %
%      Qua Inizia La Tesina        %
% ================================ %
%\abstract{
%  \lipsum[1]
%}

%!TEX TS-program = pdflatex
%!TEX root = main.tex
%!TEX encoding = UTF-8 Unicode

\section{Chap Esempio}
Qua mi piacerebbe fare una bella scarrellata di robe da cui prendere spunto e usare come riferimento.
Ma lo farò più avanti.

\cleardoublepage


%\input{./chapters/conclusione}
%\cleardoublepage


% ================================ %
%           Bibliografia           %
% ================================ %
%\cleardoublepage
%\printbibliography



\end{document}
