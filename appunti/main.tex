\documentclass{article}
%\documentclass[draft]{article}

\usepackage[italian]{babel}
\usepackage[utf8]{inputenc}


\usepackage{graphicx} % Immagini fantastiche e...
\graphicspath{        % dove trovarle
  {./images/},
  {../images/}        % Necessario se cartella chapters
}

% Bibliografia
\usepackage{csquotes}
\usepackage[
backend=biber, % backend... boh
style=numeric, % altri stili https://www.overleaf.com/learn/latex/Biblatex_bibliography_styles
sorting=nty % name, title, year % ynt % tny
]{biblatex}
\addbibresource{bibl.bib} % Il file della bibliografia

% Per blocchi di codice
\usepackage{listings}
%\lstset{ % per apici e pedici nel codice
%  mathescape
%}
\lstset{  %TODO migliorare stile
  basicstyle=\small,
  identifierstyle=,
  commentstyle=\color{gray},
  showstringspaces=false,
  stringstyle=\ttyfamily,
  numbers=left, numberstyle=\tiny, stepnumber=2, numbersep=5pt
%%%   %numbers=left,
%%%   %numberstyle=\tiny,
}


% Grafici direttamente in latex
\usepackage{tikz}
\usetikzlibrary{shapes,positioning,calc}
\colorlet{lightgray}{gray!20}

% Immagini galleggiano
%\usepackage{float}

\usepackage{caption}
% per caption ad immagini in tab annidiate
\usepackage{subcaption}

\usepackage{hyperref} % lasciare per ultimo
%\hypersetup{colorlinks=true, linkcolor=blue, citecolor=black, plainpages=false, urlcolor=blue}
\hypersetup{colorlinks=true, linkcolor=black, citecolor=black, plainpages=false, urlcolor=blue}

% Usato nella copertina
\usepackage{wallpaper}


% Simboli matematici
%\usepackage{amsthm}
\usepackage{amsfonts}
\usepackage{mathabx} % per \topdoteq
%\usepackage{mathtools, amsthm}

% Comodo per evidenziare zone da modificare
\usepackage{todonotes} 

% Lorem ipsum...
\usepackage{lipsum} 


 
% ================================ %
%          Cose Personali          %
% ================================ %
\newtheorem{theorem}{Theorem}[section]
\newtheorem{corollary}{Corollary}[theorem]
\newtheorem{lemma}[theorem]{Lemma}

% DSD Syntax Symbols
% DA USARE DENTRO MATH ENVIRONMENT
\newcommand{\toe}{^\wedge}
\newcommand{\cmpl}{^*}
\newcommand{\toec}{^{\wedge *}}

% Alcune comodita' logico/matematiche
\let\implies\rightarrow
\let\Implies\Rightarrow
\let\impliedby\leftarrow
\let\Impliedby\Leftarrow
\let\iff\Leftrightarrow

\let\et\wedge
\let\vel\vee

\newcommand\N{\ensuremath{\mathbb{N}}}
\newcommand\R{\ensuremath{\mathbb{R}}}
\newcommand\Z{\ensuremath{\mathbb{Z}}}
\renewcommand\O{\ensuremath{\emptyset}}
\newcommand\Q{\ensuremath{\mathbb{Q}}}
\newcommand\C{\ensuremath{\mathbb{C}}}

% DA USARE DENTRO NEL TESTO
% per evitare di scrivere "emph"
% notare spazio finale
\newcommand{\pattern}{\emph{pattern} }
\newcommand{\toehold}{\emph{toehold} }
\newcommand{\bind}{\emph{bind} }
\newcommand{\unbind}{\emph{unbind} }
\newcommand{\displacement}{\emph{displacement} }
\newcommand{\bond}{\emph{bond} }
\newcommand{\sg}{\emph{strand graph} }
\newcommand{\DNAcomp}{ \emph{DNA-Computing} }
\newcommand{\foot} {\emph{foot} }
\newcommand{\leg}  {\emph{leg}  }
\newcommand{\hand} {\emph{hand} }
\newcommand{\arm}  {\emph{arm}  }
% vari
%\newcommand{\lsti{args}}
%           {\lstinline{args}} % TODO cercare sintassi args



% ================================ %
%            Il Documento          %
% ================================ %
\begin{document}

% ================================ %
%        Creo Prima Pagina         %
% ================================ %

\ThisCenterWallPaper{0.95}{polloPallido}
% Intestazione
% TODO Migliorare esteticamente
\begin{titlepage}
 	\centering
  \Huge{\textbf{Approfondimento per\\ Bella Materia}}\\
 	[25mm]
  \raggedright
  \Large{\textbf{Studente:}}\\
  \Large{\textbf{Tristano Munini}}\\
 	[25mm]
  \raggedright
  \Large{\textbf{Relatore:}}\\
  \Large{\textbf{Prof. Professore}}\\
 	[80mm]
 	\centering
  \LARGE{\underline{\textbf{ANNO ACCADEMICO 2019-2020}}}\\
\end{titlepage}

% ================================ %
%              Indice              %
% ================================ %
\tableofcontents
\thispagestyle{empty}
\cleardoublepage
\setcounter{page}{1}


% ================================ %
%      Qua Inizia La Tesina        %
% ================================ %
%\abstract{
%  \lipsum[1]
%}

%!TEX TS-program = pdflatex
%!TEX root = main.tex
%!TEX encoding = UTF-8 Unicode

\section{Minizinc: 005 vs 008}
\todo[inline]{
OCCHIO!
Questa sezione è una memo per me.
Probabilmente NON verrà messa nella relazione finale
}\noindent
I due modelli sono molto diversi:
il primo si basa su una grossa matrice 3d in cui ogni cella rappresenta una stanza e mentre l'intero salvato nella cella rappresenta il tipo di ospite per quella stanza.
Il secondo modello sfrutta 4 array, la cui lunghezza dipende dal numero di persone che dobbiamo piazzare, e i valori assegnati alle celle degli array corrispondono al numero della stanza.
La formalizzazione dei vincoli è abbastanza diversa per le due codifiche.
Infatti, nella prima i vincoli di vicinanza dipendono dalle coordinate di accesso alla matrice 3d, mentre nel secondo dipendono dai valori salvati negli array.
Nel secondo modello risulta molto più facile scrivere vincoli spezza simmetrie, ad esempio l'alldifferent o il vincolo id ordinamento crescente sugli array.

Seguono una descrizione più dettagliata dei modelli ed un loro confronto.

\subsection{005}

\lstinputlisting[firstline=47,lastline=59]{code/005_covid19.mzn}



\cleardoublepage

%%!TEX TS-program = pdflatex
%!TEX root = main.tex
%!TEX encoding = UTF-8 Unicode

\section{Chap Esempio}
Qua mi piacerebbe fare una bella scarrellata di robe da cui prendere spunto e usare come riferimento.
Ma lo farò più avanti.

%\cleardoublepage


%\input{./chapters/conclusione}
%\cleardoublepage


% ================================ %
%           Bibliografia           %
% ================================ %
%\cleardoublepage
%\printbibliography



\end{document}
