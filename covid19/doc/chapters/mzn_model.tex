%!TEX TS-program = pdflatex
%!TEX root = main.tex
%!TEX encoding = UTF-8 Unicode

\section{Minizinc}
\subsection{Il modello}
%In input vengono forniti
%\lstinline{K, H, M, O, P, Q}
Similmente a quanto visto per ASP, in \emph{covid19.mzn} si ha una prima definizione delle stanze
\lstinputlisting[
firstline= 1,
lastline = 16
]{./code/covid19.mzn}

\noindent
Le limitazioni sul numero di ospiti per stanza sono date dai seguenti vincoli.
Notare come questa modellazione permette di usare \emph{alldifferent} sui numeri stanze degli ospiti in osservazione.
\lstinputlisting[
firstline= 19,
lastline = 50
]{./code/covid19.mzn}

\noindent
La relazione di \emph{Vicinato 1} è una versione più snella e comprensibile delle formule utilizzate nel modello ASP
\lstinputlisting[
firstline= 53,
lastline = 77
]{./code/covid19.mzn}

\noindent
\emph{Vicinato 2} viene soddisfatto se esiste una terza stanza su cui fare ``perno'' in \emph{Vicinato 1} con entrambe le stanze considerate
\lstinputlisting[
firstline= 80,
lastline = 87
]{./code/covid19.mzn}

\noindent
Il valore da minimizzare è calcolato allo stesso modo del codice ASP
\lstinputlisting[
firstline= 90,
lastline = 99
]{./code/covid19.mzn}

\noindent
La configurazione che ha dato i risultati migliori è riportata nel listato che segue.
Notare come si tenti fin da subito di raccogliere ospiti malati o in osservazione nelle prime stanze e sui piani più bassi, mentre per ospiti in quarantena o positivi si cerca una stanza nei piani superiori.
\lstinputlisting[
firstline= 150,
lastline = 159
]{./code/covid19.mzn}

%\cleardoublepage
\subsection{\emph{Symmetry Breaking}}
La modellazione scelta permette di implementare facilmente la rottura delle simmetrie descritta per il modello ASP
\lstinputlisting[
firstline= 103,
lastline = 113
]{./code/covid19.mzn}
