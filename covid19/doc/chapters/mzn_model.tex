%!TEX TS-program = pdflatex
%!TEX root = main.tex
%!TEX encoding = UTF-8 Unicode

\section{Minizinc}
\subsection{Il modello}
%In input vengono forniti
%\lstinline{K, H, M, O, P, Q}
Ad ogni stanza viene associato un intero partendo da 0 ed incrementando di 1.
Il numero totale di stanze è $2*K*H$, perché abbiamo $K$ corridoi e per ciascun corridoio ci sono $H$ stanze per lato (sinistro o destro).
Con questa numerazione ogni $K$ numeri si cambierà lato, mentre ogni $2*K$ numeri si indica un corridoio ad un piano superiore.
La numerazione permette di modellare abbastanza facilmente le relazioni spaziali tra stanze,
ad esempio:
le prime $H$ stanze appartengono al lato sinistro del primo corridoio;
le stanze dalla $H$ alla $2*H-1$ appartengono al lato destro del primo corridoio;
le stanze dalla $2*H$ alla $3*H-1$ sono sul secondo piano a sinistra; etc\dots
\lstinputlisting[
firstline= 1,
lastline = 16
]{./code/covid19.mzn}

\noindent
Le limitazioni sul numero di ospiti per stanza sono date dai seguenti vincoli.
Notare come questa modellazione permette di usare \emph{alldifferent} sui numeri stanze degli ospiti in osservazione.
\lstinputlisting[
firstline= 19,
lastline = 50
]{./code/covid19.mzn}

\noindent
La relazione di \emph{Vicinato 1} verifica se due stanze sono adiacenti per una delle cinque direzioni (sopra, sotto, destra, sinistra e di fronte).
\lstinputlisting[
firstline= 53,
lastline = 77
]{./code/covid19.mzn}

\noindent
Due camere sono a distanza \emph{Vicinato 2} se condividono una camera a distanza \emph{Vicinato 1}.
In pratica, presa una camera, un elemento nel suo \emph{Vicinato 2} può essere raggiunto in due passi selezionando prima un \emph{Vicinato 1} adeguato e poi un \emph{Vicinato 1} della camera appena selezionata.
Nel vincolo si richiede l'esistenza di questa terza camera che faccia da ``perno'' per lo spostamento.
%\emph{Vicinato 2} viene soddisfatto se esiste una terza stanza su cui fare ``perno'' in \emph{Vicinato 1} con entrambe le stanze considerate
\lstinputlisting[
firstline= 80,
lastline = 87
]{./code/covid19.mzn}

\noindent
%%Il valore da minimizzare è calcolato allo stesso modo del codice ASP
%La funzione da minimizzare viene calcolata contando che ospiti di che stanze soddisfano la relazione \emph{scomodo}.
%Due ospiti non possono soddisfare la relazione se occupano stanze non in vicinato tra loro, oppure se appartengono a tipologie non problematiche (es. in osservazione).
Si vuole minimizzare un intero $c$ calcolato contando tutte le coppie di ospiti con tipologia scomoda (malato, quarantena precauzionale, positivo) in \emph{Vicinato 1} oppure \emph{Vicinato 2}, a seconda della tipologia.
%In particolare si vuole che i malati non siano i
Notare che $c$ ha un valore più grande se nelle stanze in vicinato ci sono più ospiti, come si vede in \autoref{fig:esempio_1}, in cui $c=4$ a causa dei malati nella stanza $1$ in \emph{Vicinato 2} con gli ospiti in quarantena della stanza $7$.
Le stanze si contano partendo da quella in alto a sinistra, stanza numero 0 e prima stanza a sinistra del primo corridoio, e procedendo per righe.
Ogni due righe si sale al corridoio superiore, come indicato dai numeri sulla destra.
\lstinputlisting[
firstline= 90,
lastline = 99
]{./code/covid19.mzn}
\begin{figure}[ht]
  \centering
  \includegraphics[width=.8\textwidth]{esempio_1}
  \caption{Schermata di \emph{in\_out\_visualizer.py}}
  \label{fig:esempio_1}
\end{figure}

\noindent
La configurazione che ha dato i risultati migliori è riportata nel listato che segue.
Notare come si tenti fin da subito di raccogliere ospiti malati o in osservazione nelle prime stanze e sui piani più bassi, mentre per ospiti in quarantena o positivi si cerca una stanza nei piani superiori.
\lstinputlisting[
firstline= 150,
lastline = 159
]{./code/covid19.mzn}

%\cleardoublepage
\subsection{\emph{Symmetry Breaking}}
Osservando che le camere all'inizio ed alla fine dei corridoi hanno un numero inferiore di camere in \emph{Vicinato 1} (e quindi anche in \emph{Vicinato 2}), risulta sensato fissare il primo malato nella prima stanza disponibile, ossia nella prima stanza del primo corridoio.
Inoltre, poiché non ci sono differenze tra malati dello stesso tipo è possible fissare un ordinamento arbitrario.
In questo caso ad ospiti con un indice inferiore vengono assegnate camera ai piani più bassi.
\lstinputlisting[
firstline= 103,
lastline = 113
]{./code/covid19.mzn}
