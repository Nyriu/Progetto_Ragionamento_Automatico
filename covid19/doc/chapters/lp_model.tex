%!TEX TS-program = pdflatex
%!TEX root = main.tex
%!TEX encoding = UTF-8 Unicode

\section{ASP}
\subsection{Il modello}
%In input vengono forniti
%\lstinline{K, H, M, O, P, Q}

Similmente a quanto visto per Minizinc, in \emph{covid19.lp} si ha una prima definizione delle stanze
\lstinputlisting[firstline= 7, lastline = 7]{./code/covid19.lp}

\noindent
Poiché gli ospiti devono essere assegnati esattamente ad una stanza, si definisce il vincolo
\lstinputlisting[
firstline= 16,
lastline = 16
]{./code/covid19.lp}
in cui \emph{soggetto(P,T)} indica il numero identificativo $P$ associato ad un ospite e la sua tipologia:
0 significa malato;
1 significa positivo;
2 significa osservazione;
3 significa quarantena.
I predicati vengono forniti nell'input e si suppone siano corretti, quindi non è possibile che un $P$ soddisfi la relazione con due $T$ differenti.
\\
\noindent
Poiché le stanze hanno capacità limitate, che dipendono dal tipo di ospite, risulta necessario definire i vincoli
\lstinputlisting[
firstline= 18,
lastline = 25
]{./code/covid19.lp}

\noindent
Inoltre una stanza può essere condivisa solo da ospiti dello stesso tipo,
quindi vengono vietate tutte le coppie illecite
\lstinputlisting[
firstline= 28,
lastline = 28
]{./code/covid19.lp}

\noindent
Due camere sono a distanza \emph{Vicinato 1} se rispettano uno dei cinque vincoli di vicinanza.
Per motivi di efficienza vengono calcolati soltanto le vicinanze tra ospiti che danno un contributo nel calcolo della funzione di costo, quindi malati, quarantena e positivi.
Qui vengono riportati i primi due vincoli
\lstinputlisting[
firstline= 31,
lastline = 51
]{./code/covid19.lp}
%\todo{spiegare la matematica?}

\noindent
Due camere sono a distanza \emph{Vicinato 2} se condividono una camera a distanza \emph{Vicinato 1}.
Diversamente da quanto fatto in Minizinc, qui è stato definito un vincolo per ciascuno dei $12$ possibili vicini 2 di una stanza, considerando soltanto le coppie di stanze contenenti malati od ospiti in quarantena.
Nei listati seguenti sono riportati due vincoli di questo tipo
\lstinputlisting[
firstline= 90,
lastline = 101
]{./code/covid19.lp}
\lstinputlisting[
firstline= 127,
lastline = 137
]{./code/covid19.lp}

\noindent
La funzione da minimizzare viene calcolata contando che ospiti di che stanze soddisfano la relazione \emph{scomodo}.
Due ospiti non possono soddisfare la relazione se occupano stanze non in vicinato tra loro, oppure se appartengono a tipologie non problematiche (es. in osservazione).
\lstinputlisting[
firstline= 234,
lastline = 235
]{./code/covid19.lp}

\subsection{\emph{Symmetry Breaking}}
Come nel modello Minizinc, il primo malato viene assegnato alla prima stanza e viene fissato un ordinamento arbitrario.
\lstinputlisting[
firstline= 229,
lastline = 232
]{./code/covid19.lp}

%\todo[inline]{commentare la configurazione migliore}
%tweety
