%!TEX TS-program = pdflatex
%!TEX root = main.tex
%!TEX encoding = UTF-8 Unicode

\section{ASP}
\subsection{Il modello}
%In input vengono forniti
%\lstinline{K, H, M, O, P, Q}

Ad ogni stanza viene associato un intero partendo da 0 ed incrementando di 1.
Il numero totale di stanze è $2*K*H$, perché abbiamo $K$ corridoi e per ciascun corridoio ci sono $H$ stanze per lato (sinistro o destro).
Con questa numerazione ogni $K$ numeri si cambierà lato, mentre ogni $2*K$ numeri si indica un corridoio ad un piano superiore.
La numerazione permette di modellare abbastanza facilmente le relazioni spaziali tra stanze,
ad esempio:
le prime $H$ stanze appartengono al lato destro del primo corridoio;
le stanze dalla $H$ alla $2*H-1$ appartengono al lato sinistro del primo corridoio;
le stanze dalla $2*H$ alla $3*H-1$ sono sul secondo piano a destra; etc\dots

\noindent
Nel file \emph{covid19.lp} le stanze vengono definite con
\lstinputlisting[firstline= 7, lastline = 7]{./code/covid19.lp}

\noindent
Poiché gli ospiti devono essere assegnati esattamente ad una stanza, si definiscono i vincoli
\lstinputlisting[
firstline= 10,
lastline = 13
]{./code/covid19.lp}

\noindent
Poiché le stanze hanno capacità limitate, che dipendono dal tipo di ospite, risulta necessario definire
\lstinputlisting[
firstline= 16,
lastline = 22
]{./code/covid19.lp}

\noindent
Inoltre una stanza può essere condivisa solo da ospiti dello stesso tipo (esclusi i positivi),
quindi vengono vietate tutte le coppie illecite
\lstinputlisting[
firstline= 25,
lastline = 32
]{./code/covid19.lp}

\noindent
Due camere sono a distanza \emph{Vicinato 1} se rispettano uno dei vincoli tra
\lstinputlisting[
firstline= 36,
lastline = 62
]{./code/covid19.lp}
\todo{spiegare la matematica?}

\noindent
Due camere sono a distanza \emph{Vicinato 2} se condividono una camera a distanza \emph{Vicinato 1}.
In pratica, presa una camera, un elemento nel suo \emph{Vicinato 2} può essere raggiunto in due passi selezionando prima un \emph{Vicinato 1} adeguato e poi un \emph{Vicinato 1} della camera appena selezionata.
Nel vincolo si richiede l'esistenza di questa terza camera che faccia da ``perno'' per lo spostamento.
\lstinputlisting[
firstline= 65,
lastline = 68
]{./code/covid19.lp}

\noindent
La funzione da minimizzare viene calcolata contando che ospiti di che stanze soddisfano la relazione \emph{scomodo}.
Due ospiti non possono soddisfare la relazione se occupano stanze non in vicinato tra loro, oppure se appartengono a tipologie non problematiche (es. in osservazione).
\lstinputlisting[
firstline= 99,
lastline = 106
]{./code/covid19.lp}
\todo{allungare?}

%\cleardoublepage
\subsection{\emph{Symmetry Breaking}}
Poiché non ci sono differenze tra malati dello stesso tipo è possible fissare un ordinamento arbitrario.
In questo caso ad ospiti con un indice inferiore vengono assegnate camera ai piani più bassi.
\lstinputlisting[
firstline= 75,
lastline = 78
]{./code/covid19.lp}

\noindent
Osservando che le camere all'inizio ed alla fine dei corridoi hanno un numero inferiore di camere in \emph{Vicinato 1} (e quindi anche in \emph{Vicinato 2}), risulta sensato fissare il primo malato nella prima stanza disponibile, ossia nella prima stanza del primo corridoio.
\todo{giustificare meglio?}
\lstinputlisting[ 
firstline= 73,
lastline = 73
]{./code/covid19.lp}



\todo[inline]{commentare la configurazione migliore}
