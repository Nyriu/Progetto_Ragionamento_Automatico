%!TEX TS-program = pdflatex
%!TEX root = main.tex
%!TEX encoding = UTF-8 Unicode

\section{ASP}
\subsection{Il modello}
%In input vengono forniti
%\lstinline{K, H, M, O, P, Q}

Similmente a quanto visto per Minizinc, in \emph{covid19.lp} si ha una prima definizione delle stanze.
Le stanze vengono definite con
\lstinputlisting[firstline= 7, lastline = 7]{./code/covid19.lp}

\noindent
Poiché gli ospiti devono essere assegnati esattamente ad una stanza, si definiscono i vincoli
\lstinputlisting[
firstline= 16,
lastline = 16
]{./code/covid19.lp}
in cui \emph{soggetto(P,T)} indica il numero identificativo $P$ associato ad un ospite e la sua tipologia:
0 significa malato;
1 significa positivo;
2 significa osservazione;
3 significa quarantena.
I predicati vengono forniti nell'input e si suppone siano corretti, quindi non è possibile che un $P$ soddisfi la relazione con due $T$ differenti.
\\
\noindent
Poiché le stanze hanno capacità limitate, che dipendono dal tipo di ospite, risulta necessario definire
\lstinputlisting[
firstline= 18,
lastline = 25
]{./code/covid19.lp}

\noindent
Inoltre una stanza può essere condivisa solo da ospiti dello stesso tipo (esclusi i positivi),
quindi vengono vietate tutte le coppie illecite
\lstinputlisting[
firstline= 28,
lastline = 28
]{./code/covid19.lp}

\noindent
Due camere sono a distanza \emph{Vicinato 1} se rispettano uno dei cinque vincoli di vicinanza.
Per motivi di efficienza vengono calcolati soltanto le vicinanze tra ospiti  \todo{TODO here}
Qui vengono riportati i primi due
\lstinputlisting[
firstline= 31,
lastline = 51
]{./code/covid19.lp}
%\todo{spiegare la matematica?}

\noindent
Due camere sono a distanza \emph{Vicinato 2} se condividono una camera a distanza \emph{Vicinato 1}.
In pratica, presa una camera, un elemento nel suo \emph{Vicinato 2} può essere raggiunto in due passi selezionando prima un \emph{Vicinato 1} adeguato e poi un \emph{Vicinato 1} della camera appena selezionata.
Nel vincolo si richiede l'esistenza di questa terza camera che faccia da ``perno'' per lo spostamento.
\lstinputlisting[
firstline= 65,
lastline = 68
]{./code/covid19.lp}

\noindent
La funzione da minimizzare viene calcolata contando che ospiti di che stanze soddisfano la relazione \emph{scomodo}.
Due ospiti non possono soddisfare la relazione se occupano stanze non in vicinato tra loro, oppure se appartengono a tipologie non problematiche (es. in osservazione).
\lstinputlisting[
firstline= 99,
lastline = 106
]{./code/covid19.lp}
\todo{allungare?}

%\cleardoublepage
\subsection{\emph{Symmetry Breaking}}
Poiché non ci sono differenze tra malati dello stesso tipo è possible fissare un ordinamento arbitrario.
In questo caso ad ospiti con un indice inferiore vengono assegnate camera ai piani più bassi.
\lstinputlisting[
firstline= 75,
lastline = 78
]{./code/covid19.lp}

\noindent
\todo{giustificare meglio?}
\lstinputlisting[ 
firstline= 73,
lastline = 73
]{./code/covid19.lp}



\todo[inline]{commentare la configurazione migliore}
