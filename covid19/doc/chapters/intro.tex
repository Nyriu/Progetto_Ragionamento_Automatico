%!TEX TS-program = pdflatex
%!TEX root = main.tex
%!TEX encoding = UTF-8 Unicode

\section{Introduzione}
\lipsum[1]
% \begin{lstlisting}
% % Isolamento persone su cociera
% %
% % Crociera con camere doppie
% % K corridoi con h camere per lato (h a dx e h a sx)
% % corridoi sono uno sopra l'altro con possibilita'
% % che i condotti d'aerazione diffondano il virus
% % N persone etichettabili con:
% %   - malate
% %   - positive sane
% %   - osservazione (possibile contatto con positivo
% %   - quarantena precauzionale
% % Possono stare assieme nella stessa stanza:
% %   - malati
% %   - positivi sani
% %   - precauzionali % Devono stare in isolamento da soli: quelli in osservazione
% %
% % Definizione di Vicinanaza tra stanze
% % Vicinato 1
% % Le camere adiacenti, quella di fronte, quelle sopra e sotto
% % Ovviamente camere agli estremi dei corridoi e quelle dei corridoi
% % 1 e k hanno meno Vicini 1
% %
% % Vicinato 2
% % Applico Vicinato 1 due volte
% %
% %
% %% Si vuole disporre le persone per avere MINIMO
% %% numero di malati a distanza <= 2 dai precauzionali e
% %%                  a distanza <= 1 dai positivi sani
% %%
% %% (eventuali vincoli extra su positivi sani)
% %
% \end{lstlisting}



%\begin{figure}[ht]
%  \centering
%  \includegraphics[width=.5\textwidth]{example-image-a}
%  \caption{Interfaccia con simulazione avviata}
%  \label{fig:ex_zombie}
%\end{figure}

%\lstinputlisting[
%firstline= 6,
%lastline = 33,
%caption={1},
%captionpos=b]{./code/covid19.mzn}
